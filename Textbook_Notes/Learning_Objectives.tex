\documentclass{article}
\usepackage[utf8]{inputenc}
\usepackage{fullpage}
\usepackage{float}
\usepackage{graphicx}
\usepackage{gensymb}
\usepackage{amsmath}

\title{ECE109 Textbook Learning Objectives}
\author{Rohit Singh}
\date{Winter 2023}

\begin{document}

\maketitle

\tableofcontents

\section{Objectives}

\subsection{Chapter 1}

\begin{enumerate}
    \item List six different property classifications of materials that determine their applicability.
    \item Cite the four components that are involved in the design, production, and utilization of materials, and briefly describe the interrelationships between these components.
    \item Cite three criteria that are important in the materials selection process.
    \item \begin{enumerate}
        \item List the three primary classifications of solid materials, and then cite the distinctive chemical feature of each
        \item Note the four types of advanced materials and, for each, its distinctive feature(s)
    \end{enumerate}
    \item \begin{enumerate}
        \item Briefly define \emph{smart material/system}
        \item Briefly explain the concept of \emph{nanotechnology} as it applies to materials.
    \end{enumerate}
\end{enumerate}

\subsection{Chapter 2}

\begin{enumerate}
    \item Name the two atomic models cited, and note the differences between them
    \item Describe the important quantum-mechanical principle that relates to electron energies
    \item Schematically plot attractive, repulsive, and net energies versus interatomic seperation for two atoms or ions
    \item \begin{enumerate}
        \item Briefly describe ionic, covalent, metallic, hydrogen, and van der Waals bonds.
        \item Note which materials exhibit each of these bonding types.
    \end{enumerate}
\end{enumerate}

\subsection{Chapter 3}

\begin{enumerate}
    \item Describe the difference in atomic/molecular structure between crystalline and non-crystalline materials
    \item Draw unit cells for face-centered cubic, body-centered cubic, and hexagonal close-packed crystal structures
    \item Derive the relationships between unit cell edge length anbd atomic radius for face-centered cubic and body-centered cubic crystal structures
    \item Compute the densities for metals having face-centered cubic and body-centered cubic crystal structures given their unit cell dimensions
    \item Given three direction index integers, sketch the direction corresponding to these indices within a unit cell
    \item Specify the Miller indices for a plane that has been drawn within a unit cell
    \item Describe how face-centered cubic and hexagonal close-packed crystal structures may be generated by stacking of close-packed planes of atoms
    \item Distinguish between single crystals and poly-crystalline materials
    \item Define \textit{isotropy} and \textit{anisotropy} with respect to material properties
\end{enumerate}

\section{Answers}

\subsection{Chapter 1}

\begin{enumerate}
    \item The important properties of materials can be grouped into six different categories (DOT-MEM): \begin{itemize}
    \item \textbf{Deteriorative}: Characteristics relating to the chemical reactivity of materials. Ex: Corrosion resistance of metals
    \item \textbf{Optical}: Characteristics describing the stimulus response of a material to electromagnetic or light radiation. Ex: Index of refraction and reflectivity
    \item \textbf{Thermal}: Characteristics describing the stimulus response of a material to changes in temperature or temperature gradients across a material. Ex: Thermal expansion and heat capacity
    \item \textbf{Mechanical}: Characteristics describing the stimulus response of a material to deformation or an applied load/force. Ex: Stiffness, strength, resistance to fracture and more.
    \item \textbf{Electrical}: Characteristics describing the stimulus response of a material to an applied electric field. Ex: Electrical conductivity and dielectric constant
    \item \textbf{Magnetic}: The responses of a material to the application of a magnetic field. Ex: Magnetic susceptibility and magnetization
\end{itemize}

\item The four components of the discipline of materials science and engineering are: \begin{itemize}
    \item \textbf{Processing}
    \item \textbf{Structure}
    \item \textbf{Properties}
    \item \textbf{Performance}
\end{itemize}

\item The three important criteria in materials selection are:
\begin{itemize}
    \item In-service conditions to which the material will be subjected
    \item Deterioration of material properties during operation
    \item Economics or cose of the fabricated piece
\end{itemize}

\item \begin{enumerate}
    \item The three primary classifications of solid materials are:
    \begin{enumerate}
        \item \textbf{Metals}: Highly-conductive to electricity and heat, strong and pliable
        \item \textbf{Ceramics}: Highly-heat resistant, various optical properites.
        \item \textbf{Polymers}: Highly-insulative to electricity and heat, pliable and environment-resistant
    \end{enumerate}
    \item The four types of advanced materials are:
    \begin{enumerate}
        \item \textbf{Semiconductors}: Conductivity that can be finely controlled with the presence of doping atoms to achieve desired characteristics
        \item \textbf{Biomaterials}: Materials that can be implanted to organic bodies to extend life or heal.
        \item \textbf{Smart Materials}: Materials that have some sensor/actuator pair that allow the material to respond to changes in its environment
        \item \textbf{Nano-technologies}: Materials that are distinguished by their size
    \end{enumerate}
\end{enumerate}
\item See the above answer
\end{enumerate}

\subsection{Chapter 2}

\begin{enumerate}
    \item The two atomic models cited are:
    \begin{enumerate}
        \item \textbf{Bohr's Model of the Atom}: Theorizes that electrons orbit the nucleus in orbitals and that their position is relatively well-known, and that electrons can jump between discrete/quantized energy levels by eiher absorbing or emitting energy, to move to a higher or lower energy level, respectively.
        \item \textbf{Wave-mechanical model}: Theorizes that electrons do not orbit the nucleus in orbitals, but instead stipulates the existance of electron clouds in which an electron is guaranteed to exist, although the exact position is represented using a probability distribution function.
    \end{enumerate}
    \item Describe the important quantum-mechanical principle that relates to electron energies
    \begin{itemize}
        \item The energies of electrons are quantized, meaning that they can only take on specific values of energy, either by jumping to a higher energy level and absorbing energy, or jumping to a lower energy level and emitting energy.
    \end{itemize}
    \item \begin{enumerate}
        \item Briefly describe ionic, covalent, metallic, hydrogen, and van der Waals bonds.
        \begin{itemize}
            \item \textbf{Ionic}: In ionic bonding, metallic and nonmetallic atoms are bonded as the latter gives up valence electrons to the former, causing the metal to become positively charged and the nonmetal to become negatively charged, forming a non-directional bond between the two atoms formed by the attraction of the two oppositely charged ions.
            \item \textbf{Covalent}: In covalent bonding, two atoms of similar electronegativity share a valence electron, causing the electron to belong to both atoms, resulting in bonding between the two atoms as they are both attached to the shared electrons.
            \item \textbf{Metallic}: In metallic bonding, an ion core formed of nuclei is bonded together while the electrons belonging to the individual nuclei are instead owned by the entire metal, resulting in a "sea of electrons" that can flow easily, resulting in improved electric and thermal conduction, with the ion core acting as the glue for the sea of electrons.
            \item \textbf{Hydrogen}: In hydrogen bonding, (a special case of van der Waals bonding), highly polar molecules form when hydrogen covalently bonds to a non-metallic element such as flourine.
            \item \textbf{van der Waals}: In van der Waals bonding, atoms are bonded as the result of electric dipoles (either induced or permanent).
        \end{itemize}
        \item Note which materials exhibit each of these bonding types.
        \begin{itemize}
            \item See the above answer
        \end{itemize}
    \end{enumerate}
\end{enumerate}

\subsection{Chapter 3}

\begin{enumerate}
    \item Describe the difference in atomic/molecular structure between crystalline and non-crystalline materials
    \item Draw unit cells for face-centered cubic, body-centered cubic, and hexagonal close-packed crystal structures
    \item Derive the relationships between unit cell edge length and atomic radius for face-centered cubic and body-centered cubic crystal structures
    \item Compute the densities for metals having face-centered cubic and body-centered cubic crystal structures given their unit cell dimensions
    \item Given three direction index integers, sketch the direction corresponding to these indices within a unit cell
    \item Specify the Miller indices for a plane that has been drawn within a unit cell
    \item Describe how face-centered cubic and hexagonal close-packed crystal structures may be generated by stacking of close-packed planes of atoms
    \item Distinguish between single crystals and poly-crystalline materials
    \item Define \textit{isotropy} and \textit{anisotropy} with respect to material properties
\end{enumerate}

\end{document}