\documentclass{article}
\usepackage[utf8]{inputenc}
\usepackage{fullpage}
\usepackage{float}
\usepackage{graphicx}
\usepackage{gensymb}
\usepackage{amsmath}

\title{ECE109 Textbook Chapter 3 Crystalline Solid Structure Notes}
\author{Rohit Singh}
\date{Winter 2023}

\begin{document}

\maketitle

\tableofcontents

\section{Learning Objectives}

\begin{enumerate}
    \item Describe the difference in atomic/molecular structure between crystalline and non-crystalline materials
    \item Draw unit cells for face-centered cubic, body-centered cubic, and hexagonal close-packed crystal structures
    \item Derive the relationships between unit cell edge length and atomic radius for face-centered cubic and body-centered cubic crystal structures
    \item Compute the densities for metals having face-centered cubic and body-centered cubic crystal structures given their unit cell dimensions
    \item Given three direction index integers, sketch the direction corresponding to these indices within a unit cell
    \item Specify the Miller indices for a plane that has been drawn within a unit cell
    \item Describe how face-centered cubic and hexagonal close-packed crystal structures may be generated by stacking of close-packed planes of atoms
    \item Distinguish between single crystals and poly-crystalline materials
    \item Define \textit{isotropy} and \textit{anisotropy} with respect to material properties
\end{enumerate}

\section{Crystal Structures}

\subsection{Fundamental Concepts}

%DEF_START{Crystalline Material}
\textbf{Crystalline materials} are those in which the atoms are situated in a repeating or periodic array over large atomic distances, that is, long-range order exists such that upon solidification, atoms will position themselves in a repetitive 3D pattern in which each atom is bonded to its nearest neighbour atoms.
%DEF_END{Crystalline Material}

All metals, many ceramics and certain polymers form crystalline structures under normal solidification conditions. Many of the properties for these crystalline structures depends on the manner in which atoms, ions or molecules are spatially arranged. Often, the term \textbf{lattice} is used to describe crystal structures, which refers to a 3D array of points coinciding with atom positions or sphere centers.

\subsection{Unit Cells}

%DEF_START{Unit Cells}
Since the atomic order in crystal structures is a repetitive pattern, it can be useful to consider the entire structure as a combination of the smallest unit that describes the repeating pattern, or the \textbf{unit cell}. The unit cell is the basic structural unit or building block of the structure, and should be defined in a way that represents the symmetry of the crystal structure.
%DEF_END{Unit Cells}

\subsection{Metallic Crystal Structures}

Two important characteristics of crystal structures are \textbf{coordination number} and \textbf{atomic packing factor (APF)}. 

\begin{itemize}
    \item \textbf{Coordination number}: The number of nearest neighbour or touching atoms
    \item \textbf{Atomic packing factor}: The sum of the sphere volumes of all atoms within a unit cell divided by the unit cell volume: $\text{APF} = \frac{\text{Volume of atoms in unit cell}}{\text{Total unit cell volume}}$
\end{itemize}

\subsubsection{Face-Centered Cubic Crystal Structure}

%DEF_START{Face-Centered Cubic}
\textbf{Face-Centered Cubic} crystal structures are those which have a cubic unit cell geometry with atoms located at each of the corners and centers of all the cube faces. The ion cores within this structure touch each other across a face diagonal, with the cube edge length $a$ and the atomic radius $R$ related by equation \ref{eq:fccA}

\begin{equation}\label{eq:fccA}
    a = 2R\sqrt{2}
\end{equation}

The FCC crystal structure has the following properties:

\begin{itemize}
    \item Eight corner atoms, 6 face atoms and no interior atoms, giving it a total of 4 atoms per unit cell.
    \item Coordination number of 12 since the front face atom has four corner nearest-neighbour atoms surrounding it, four face atoms in contact from behind and four other equivalent face atoms in the next unit cell (4+4+4=12).
    \item APF of 0.74, which is the maximum packing possible for spheres all having the same diameter. 
\end{itemize}
%DEF_END{Face-Centered Cubic}

\subsubsection{Body-Centered Cubic Crystal Structure}

%DEF_START{Body-Centered Cubic Crystal Structure}
\textbf{Body-Centered Cubic (BCC)} crystal structures are those which have a cubic unit cell geometry with atoms located at each of the corners with a single atom at the cube center. The ion cores within this structure touch each other along cube diagonals, with the unit cell length $a$ and atomic radius $R$ related by equation \ref{eq:bccA}

\begin{equation}\ref{eq:bccA}
    a = \frac{4R}{\sqrt{3}}
\end{equation}

The BCC crystal structure has the following properties:

\begin{itemize}
    \item Eight corner atoms, one interior atom, giving it a total of 2 atoms per unit cell.
    \item Coordination number of 8, since each centre atom has as nearest neighbour 8 corner atoms
    \item APF of 0.68
\end{itemize}
%DEF_END{Body-Centered Cubic Crystal Structure}

\subsection{Density Computations}

The theoretical density of a crystal structure of a solid can be found using equation \ref{eq:density}

\begin{equation}\label{eq:density}
    \rho = \frac{nA}{V_C N_A}
\end{equation}

Where: 
\begin{itemize}
    \item $n$ = Number of atoms associated with each unit cell
    \item $A$ = Atomic weight
    \item $V_C$ = Volume of the unit cell
    \item $N_A$ = Avogadro's number ($6.022 \times 10^{23} \text{atoms mol}^{-1}$)
\end{itemize}

\subsection{Polymorphism and Allotropy}

%DEF_START{Polymorphism,Allotropy}
Metals (and non-metals) which have more than one crystal structure are referred to as \textbf{polymorphic}. When this phenomena is observed in elemental solids, it is referred to as \textbf{allotropy}. The prevailing crystal structure depends on both the temperature and external pressure. An example of a polymorphic material is carbon, specifically graphite.
%DEF_END{Polymorphism,Allotropy}

\subsection{Crystal Systems}

To categorize the number different possible crystal structures, we divide them into groups according to unit cell configurations and/or atomic arrangements. One scheme to achieve this is based on the unit-cell geometry, in which a 3D coordinate system is established with its origin at one of the unit cell corners, and the $x,y,z$ axes coinciding with one of the three parallelepiped edges that extend from this corner. From this, the unit cell geometry can be completely defined in terms of six parameters, the three edge lengths ($a,b,c$) and the three interaxial angles ($\alpha, \beta, \gamma$). These are often referred to as the \textbf{lattice parameters} of the crystal structure.

There are seven different possible combinations of the lattice parameters, each of which represents a distinct \textbf{crystal system}, which are shown in table \ref{tab:latticeParamRels}:

\begin{table}[h]
    \centering
    \begin{tabular}{|c|c|c|}
         \hline
         \textbf{Crystal System} & \textbf{Axial Relationships} & \textbf{Interaxial Angles} \\
         \hline
         \textbf{Cubic} & $a=b=c$  & $\alpha = \beta = \gamma = 90^\degree$  \\ 
         \textbf{Hexagonal} & $a=b \ne c$ & $\alpha = \beta = 90^{\degree}, \gamma = 120^\degree$ \\
         \textbf{Tetragonal} & $a=b \ne c$ & $\alpha = \beta = \gamma = 90^\degree$ \\
         \textbf{Rhombohedral} & $a=b=c$ & $\alpha = \beta = \gamma \ne 90^\degree$ \\
         \textbf{Orthorhombic} & $a \ne b \ne c$ & $\alpha = \beta = \gamma = 90^\degree$ \\ 
         \textbf{Monoclinic} & $a \ne b \ne c$ & $\alpha = \gamma = 90^{\degree} \ne \beta$ \\
         \textbf{Triclinic} & $a \ne b \ne c$ & $\alpha \ne \beta \ne \gamma \ne 90^\degree$ \\
         \hline
    \end{tabular}
    \caption{Lattice Parameter Relationships}
    \label{tab:latticeParamRels}
\end{table}


\begin{itemize}

\end{itemize}

\end{document}